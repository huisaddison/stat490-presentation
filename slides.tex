\input{sty/header.tex}
\newcommand{\blueone}{{\color{yaleblue} 1}}
\newcommand{\greyzero}{{\color{solarized@base00} 0}}
\usepackage{sty/personalmacros}
\usepackage{sty/personalslides}

\begin{document}
%%%%%%%%%%%%%%%%%%%%%%%%%%%%%%%%%%%%%%%%%%%%%%%%%%%%%%%%%%%%%%%%%%%%%%%%%%%%%%%
% Manifest
% --------
% - Title
% - Outline
% - Review of Graphical Models
% - Review of Multivariate Gaussian
% - Precision Matrices in Context of Gaussian Graphical Models
% - Pairwise Inference for Entrywise Recovery
% - Upper Bound ($\ell_infty$ norm)
%   - Statement of Theorems 2 and 3
%   - Proof / Intuition for Theorems 2 and 3
% - Lower Bound ($\ell_infty$ norm)
%   - Statement of Theorem 5
%   - Proof / Intuition for Theorem 5
% - Next Steps
%%%%%%%%%%%%%%%%%%%%%%%%%%%%%%%%%%%%%%%%%%%%%%%%%%%%%%%%%%%%%%%%%%%%%%%%%%%%%%%
\begin{frame}[fragile] \frametitle{}
\vfill
\vspace{0.2cm}
{
    \color{yaleblue}
    \fontsize{0.5cm}{0cm}\selectfont
    Midterm Presentation: \\
}
\vspace{1.0cm}
{
    \fontsize{0.7cm}{0cm}\selectfont
    Risk Properties in Bandable Precision Matrix Estimation\\
}

\hfill

\vspace{1.8cm}
\begin{minipage}{1.0\textwidth}\raggedleft
    \color{yaleblue}
    Addison Hu   \\
    Statistics 490 \\
    01 March 2017
\end{minipage}
\end{frame}
%%%%%%%%%%%%%%%%%%%%%%%%%%%%%%%%%%%%%%%%%%%%%%%%%%%%%%%%%%%%%%%%%%%%%%%%%%%%%%%
% Outline
%%%%%%%%%%%%%%%%%%%%%%%%%%%%%%%%%%%%%%%%%%%%%%%%%%%%%%%%%%%%%%%%%%%%%%%%%%%%%%%
\begin{frame}[fragile] \frametitle{}
    \slideheader{Outline}
    \begin{enumerate}
        \item Refresher on Graphical Models \& Multivariate Gaussian
        \item Pairwise Inference for Entrywise Recovery of $\Sigma^{-1}$
        \item Risk Bounds for Entrywise Recovery in $\norm{\cdot}_\infty$
        \item Next Steps
    \end{enumerate}
\end{frame}
%%%%%%%%%%%%%%%%%%%%%%%%%%%%%%%%%%%%%%%%%%%%%%%%%%%%%%%%%%%%%%%%%%%%%%%%%%%%%%%
% Refresher
%%%%%%%%%%%%%%%%%%%%%%%%%%%%%%%%%%%%%%%%%%%%%%%%%%%%%%%%%%%%%%%%%%%%%%%%%%%%%%%
\begin{frame}[fragile] \frametitle{}
    \sectionslide{Refresher}
\end{frame}
%%%%%%%%%%%%%%%%%%%%%%%%%%%%%%%%%%%%%%%%%%%%%%%%%%%%%%%%%%%%%%%%%%%%%%%%%%%%%%%
\begin{frame}[fragile] \frametitle{}
    \slideheader{Graphical Models}
    \begin{itemize}
        \item Graphical models provide a framework within which to consider
            dependence structure within a group of variables.
        \item In doing so, we may relax the i.i.d. assumption and still perform
            inference feasibly.
        \item Examples:
            \begin{itemize}
                \item Facebook users graph
                \item Gene interaction networks
            \end{itemize}
    \end{itemize}
\end{frame}
%%%%%%%%%%%%%%%%%%%%%%%%%%%%%%%%%%%%%%%%%%%%%%%%%%%%%%%%%%%%%%%%%%%%%%%%%%%%%%%
\begin{frame}[fragile] \frametitle{}
    \slideheader{Markov Random Fields}
    \begin{itemize}
        \item Consider a graph $G = (V, E)$, and a corresponding set of
            random variables $\{X_i\}_{i=1}^{|V|}$, where the random variables
            are indexed by $u \in V$.
        \item \textbf{Pairwise Markov property:} $X_u \indep X_v | X_{V
            \setminus\{u, v\}}$ for any two non-adjacency nodes $u, v$.
        \item \textbf{Local Markov property:} $X_u \indep X_{V \setminus
            \mathrm{cl}(u)} | X_{\mathrm{nb}(u)}$ for any node $v$.
        \item \textbf{Global Markov property:} $X_A \indep X_B | X_S$ for
            disjoint $A, B \subset V$, and a separating subset $S$.
        \item Inference is easy when the edges are known; but is more
            interesting when they are unknown.
    \end{itemize}
    % https://en.wikipedia.org/wiki/Markov_random_field
\end{frame}
%%%%%%%%%%%%%%%%%%%%%%%%%%%%%%%%%%%%%%%%%%%%%%%%%%%%%%%%%%%%%%%%%%%%%%%%%%%%%%%
\begin{frame}[fragile] \frametitle{}
    \slideheader{Example: Hub and Spoke Model}
    \vspace{1cm}
    \begin{columns}[T]
        \begin{column}{.48\textwidth}
            $$
            \bmat{
                \blueone & \greyzero & \greyzero & \blueone & \greyzero &
                \greyzero & \greyzero   \\ 
                \greyzero & \blueone & \greyzero & \blueone & \greyzero &
                \greyzero & \greyzero   \\ 
                \greyzero & \greyzero & \blueone & \blueone & \greyzero &
                \greyzero & \greyzero   \\ 
                \blueone & \blueone & \blueone & \blueone & \blueone & \blueone
                & \blueone   \\ 
                \greyzero & \greyzero & \greyzero & \blueone & \blueone &
                \greyzero & \greyzero   \\ 
                \greyzero & \greyzero & \greyzero & \blueone & \greyzero &
                \blueone & \greyzero   \\ 
                \greyzero & \greyzero & \greyzero & \blueone & \greyzero &
                \greyzero & \blueone
            }
            $$
        \end{column}
        \hfill
        \begin{column}{.48\textwidth}
            \vspace{-0.2cm}
            \hspace{-0.5cm}
            \includegraphics[scale=0.25]{img/spokes.png}
        \end{column}
    \end{columns}
\end{frame}
%%%%%%%%%%%%%%%%%%%%%%%%%%%%%%%%%%%%%%%%%%%%%%%%%%%%%%%%%%%%%%%%%%%%%%%%%%%%%%%
\begin{frame}[fragile] \frametitle{}
    \slideheader{Multivariate Gaussian}

    Suppose $X \dist \Nn(\mu, \Sigma)$. Its density function is given by:
	$$
    p(\xx) = \left(2\pi\right)^{-\frac{p}{2}}\left|\Sigma\right|^{-\frac{1}{2}}
    \exp\lc-\frac{1}{2}(\xx - \mathbf{\mu})^\top \Sigma^{-1}
    (\xx - \mathbf{\mu})\rc
    $$
    \begin{itemize}
        \item Closure properties:
            \begin{itemize}
                \item Sum of independent Gaussian random variables is Gaussian.
                \item Marginal of a joint Gaussian distribution is Gaussian.
                \item Condition of a joint Gaussian distribution is Gaussian.
            \end{itemize}
        \item The sparsity pattern of $\Sigma^{-1}$ concides with the adjacency
            matrix of the associated MRF.
    \end{itemize}
    % http://cs229.stanford.edu/section/more_on_gaussians.pdf
\end{frame}
%%%%%%%%%%%%%%%%%%%%%%%%%%%%%%%%%%%%%%%%%%%%%%%%%%%%%%%%%%%%%%%%%%%%%%%%%%%%%%%
\begin{frame}[fragile] \frametitle{}
    \slideheader{Multivariate Gaussian, cont.}

	\begin{itemize}
        \item Closure under marginalization:  Suppose $A \subset V$.  Then
            $$
            \Sigma_{A} = \lp\Sigma_{ij}\rp_{i\in A, j \in A}
            $$
        \item Closure under conditioning: Suppose  $A, B \subset V$, $A \cup B
            = V, A \cap B = \varnothing$.  Then:
            \begin{align*}
                \lp \Omega_A \rp ^{-1}  &= \Sigma_{A|B}   \\
                \lp \Sigma_A \rp ^{-1}  &= \Omega_{A|B}   \\
            \end{align*}
	\end{itemize}
\end{frame}
%%%%%%%%%%%%%%%%%%%%%%%%%%%%%%%%%%%%%%%%%%%%%%%%%%%%%%%%%%%%%%%%%%%%%%%%%%%%%%%
% Pairwise Recovery
%%%%%%%%%%%%%%%%%%%%%%%%%%%%%%%%%%%%%%%%%%%%%%%%%%%%%%%%%%%%%%%%%%%%%%%%%%%%%%%
\begin{frame}[fragile] \frametitle{}
    \sectionslide{Precision Matrix Estimation}
\end{frame}
%%%%%%%%%%%%%%%%%%%%%%%%%%%%%%%%%%%%%%%%%%%%%%%%%%%%%%%%%%%%%%%%%%%%%%%%%%%%%%%
\begin{frame}[fragile] \frametitle{}
    \slideheader{Maximum Likelihood Estimation}

    Assume $\mu = 0$.  Then the maximum likelihood estimation problem is:
	\begin{align*}
		\maximize{
			-\log\det|\Sigma| - \langle \hat\Sigma, \Sigma^{-1} \rangle
		}{\Sigma}{\Sigma \gengeq 0}
	\end{align*}

    \begin{itemize}
        \item Maximum Likelihood Estimate given by $\hat\Sigma = 
            \frac{1}{n} \XX^\top \XX$.
        \item Idea: $\hat\Omega = \hat\Sigma^{-1}$.
        \item Issues:
            \begin{itemize}
                \item Invertibility \& Conditioning
                \item Noise \& Sparsity
            \end{itemize}
    \end{itemize}
\end{frame}
%%%%%%%%%%%%%%%%%%%%%%%%%%%%%%%%%%%%%%%%%%%%%%%%%%%%%%%%%%%%%%%%%%%%%%%%%%%%%%%
\begin{frame}[fragile] \frametitle{}
    \slideheader{Graphical Lasso}
\end{frame}
%%%%%%%%%%%%%%%%%%%%%%%%%%%%%%%%%%%%%%%%%%%%%%%%%%%%%%%%%%%%%%%%%%%%%%%%%%%%%%%
\begin{frame}[fragile] \frametitle{}
    \slideheader{Asymptotic Normal Thresholding (ANT)}
\end{frame}
%%%%%%%%%%%%%%%%%%%%%%%%%%%%%%%%%%%%%%%%%%%%%%%%%%%%%%%%%%%%%%%%%%%%%%%%%%%%%%%
% Risk Bounds
%%%%%%%%%%%%%%%%%%%%%%%%%%%%%%%%%%%%%%%%%%%%%%%%%%%%%%%%%%%%%%%%%%%%%%%%%%%%%%%
\begin{frame}[fragile] \frametitle{}
    \sectionslide{Risk Bounds in $\norm{\cdot}_\infty$}
\end{frame}
%%%%%%%%%%%%%%%%%%%%%%%%%%%%%%%%%%%%%%%%%%%%%%%%%%%%%%%%%%%%%%%%%%%%%%%%%%%%%%%
\begin{frame}[fragile] \frametitle{}
    \slideheader{Risk Upper Bound}

    \theorem{Lorem ipsum.}

\end{frame}
%%%%%%%%%%%%%%%%%%%%%%%%%%%%%%%%%%%%%%%%%%%%%%%%%%%%%%%%%%%%%%%%%%%%%%%%%%%%%%%
\begin{frame}[fragile] \frametitle{}
    \slideheader{Oracle Inequalities}
\end{frame}
%%%%%%%%%%%%%%%%%%%%%%%%%%%%%%%%%%%%%%%%%%%%%%%%%%%%%%%%%%%%%%%%%%%%%%%%%%%%%%%
\begin{frame}[fragile] \frametitle{}
    \slideheader{Coupling Argument}
\end{frame}
%%%%%%%%%%%%%%%%%%%%%%%%%%%%%%%%%%%%%%%%%%%%%%%%%%%%%%%%%%%%%%%%%%%%%%%%%%%%%%%
\begin{frame}[fragile] \frametitle{}
    \slideheader{Risk Lower Bound}
\end{frame}
%%%%%%%%%%%%%%%%%%%%%%%%%%%%%%%%%%%%%%%%%%%%%%%%%%%%%%%%%%%%%%%%%%%%%%%%%%%%%%%
\begin{frame}[fragile] \frametitle{}
    \slideheader{Le Cam's Two-Point Argument}
\end{frame}
%%%%%%%%%%%%%%%%%%%%%%%%%%%%%%%%%%%%%%%%%%%%%%%%%%%%%%%%%%%%%%%%%%%%%%%%%%%%%%%
\begin{frame}[fragile] \frametitle{}
    \sectionslide{Next Steps}
\end{frame}
%%%%%%%%%%%%%%%%%%%%%%%%%%%%%%%%%%%%%%%%%%%%%%%%%%%%%%%%%%%%%%%%%%%%%%%%%%%%%%%
\end{document}

